% definit le type de document et ses options
\documentclass[a4paper,11pt]{article}

% des paquetages utiles classiques, en ajouter d'autres selon vos besoins
\usepackage[utf8]{inputenc}
\usepackage[T1]{fontenc}
\usepackage{amsmath}
\usepackage{amssymb,calc}
% \usepackage{fullpage}
% \usepackage{stmaryrd}
% \usepackage{url}
% \usepackage{xspace}
\usepackage[francais]{babel}

% Pour les figures
\usepackage{graphicx}

% Pour les listes à puces
\usepackage{enumitem}

\usepackage{indentfirst}

% Évite un conflit entre french babel et enumitem
\frenchbsetup{StandardLists=true}

\usepackage[standard,framed]{ntheorem}
\usepackage{framed}

% Pour des commentaires à droite d'équations
\newenvironment{rcases}
  {\left.\begin{aligned}}
  {\end{aligned}\right\rbrace}

\newenvironment{lcases}
  {\left\lbrace\begin{aligned}}
  {\end{aligned}\right.}


\newcommand{\reff}[1]{(\ref{#1})}


\setlength{\parindent}{30pt}
\setlength{\parskip}{1ex}
\setlength{\textwidth}{15cm}
\setlength{\textheight}{24cm}
\setlength{\oddsidemargin}{0.2cm}
\setlength{\evensidemargin}{-.7cm}
\setlength{\topmargin}{-.5in}

% des commandes pratiques pour ecrire des maths :
\newcommand{\dx}{\,dx}
\newcommand{\dt}{\,dt}
\newcommand{\ito}{,\dotsc,}
\newcommand{\R}{\mathbb{R}}
\newcommand{\N}{\mathbb{N}}
\newcommand{\C}{\mathbb{C}}
\newcommand{\Z}{\mathbb{Z}}
\newcommand{\Poly}[1]{\mathcal{P}_{#1}}
\newcommand{\abs}[1]{\left\lvert#1\right\rvert}
\newcommand{\norm}[1]{\left\lVert#1\right\rVert}
\newcommand{\pars}[1]{\left(#1\right)}
\newcommand{\bigpars}[1]{\bigl(#1\bigr)}
\newcommand{\set}[1]{\left\{#1\right\}}
\newcommand{\tpo}[1]{\,^t#1}


\DeclareMathOperator{\Sup}{Sup}
\DeclareMathOperator{\Max}{Max}

%
%\lstset{
%  literate=%
%  {À}{{\`A}}1 {Â}{{\^A}}1 {Ç}{{\c{C}}}1%
%  {à}{{\`a}}1 {â}{{\^a}}1 {ç}{{\c{c}}}1%
%  {É}{{\'E}}1 {È}{{\`E}}1 {Ê}{{\^E}}1 {Ë}{{\"E}}1% 
%  {é}{{\'e}}1 {è}{{\`e}}1 {ê}{{\^e}}1 {ë}{{\"e}}1%
%  {Ï}{{\"I}}1 {Î}{{\^I}}1 {Ô}{{\^O}}1%
%  {ï}{{\"i}}1 {î}{{\^i}}1 {ô}{{\^o}}1%
%  {Ù}{{\`U}}1 {Û}{{\^U}}1 {Ü}{{\"U}}1%
%  {ù}{{\`u}}1 {û}{{\^u}}1 {ü}{{\"u}}1%
%}

\newlength\dlf
\newcommand\alignedbox[2]{
  % #1 = before alignment
  % #2 = after alignment
    &
    \begingroup
    \settowidth\dlf{$\displaystyle #1$}
    \addtolength\dlf{\fboxsep+\fboxrule}
    \hspace{-\dlf}
    \boxed{#1 #2}
\endgroup
}

%%%%% PACKAGE AMSTHM :
% \newtheoremstyle{definition}
%   {20pt}   % ABOVESPACE
%   {20pt}   % BELOWSPACE
%   {}  % BODYFONT
%   {0pt}       % INDENT (empty value is the same as 0pt)
%   {\bfseries} % HEADFONT
%   {.}         % HEADPUNCT
%   {5pt plus 1pt minus 1pt} % HEADSPACE
%   {}          % CUSTOM-HEAD-SPEC

\newframedtheorem{ftheo}[theorem]{Theoreme}

\theoremstyle{plain} % default
\theorembodyfont{\normalfont}
\newframedtheorem{fdef}[definition]{Définition}

\renewtheorem{remark}{Remarque}

\newframedtheorem{coroll}{Corollaire}

\newtheorem{rappel}{Rappel}

\newtheorem{preuve}{Preuve}

\title{\huge \bfseries Factorisation de Cholesky}
\date{}


\begin{document}
\maketitle

RECOPIER L'INTRO

$A$ symétrique et définie positive (def > 0)

\begin{remark}
    $A$ est symérique et définie positive si et seulement si :
    \begin{enumerate}
        \item $^tA = A$
        \item $^tXAX \geq 0, \forall x \in \R$
        \item $^tXAX = 0 \Rightarrow X = 0$
    \end{enumerate}

    $A$ définie positive $\Rightarrow$ $A$ inversible
    $A$ symétrique $\Rightarrow$ $A$ diagonalisable. $\exists P / A = P^{-1}DP$
    $A$ symétrique et définie positive $\Rightarrow$ toutes les valeurs propres de $A$ sont positives et réelles.
\end{remark}

\begin{ftheo}
    Soit $A \in M_n(\R)$ symétrique définie positive. Il existe (au moins) une matrice réelle triangulaire inférieure $T$ telle que :
    \[
        A = T \,^tT
    \]
    De plus, si on impose que les éléments diagonaux de T soient tous positifs, alors la factorisation $A = T \,^tT$ est unique.
\end{ftheo}


\begin{preuve}[Existence]
    Notons $\Delta_k$ la sous-matrice d'éléments $a_{ij}, 1 \leq i,j \leq k$.
    $\Delta_k$ sont inversibles car elles sont symétriques et définies positives.
    Donc d'après le théorème 1 du précédent chapitre :
    \[
        A = LU \text{ avec } L =
        \begin{pmatrix}
            1 & 0      & \dots  & 0 \\
            - & \ddots & \dots  & 0 \\
            - & -      & \ddots & 0 \\
            - & -      & -      & 1 \\
        \end{pmatrix}
    \]
    trou
\end{preuve}


Notons $D=diag(u_{ij})$. On a  $A=\tpo A = \tpo U \tpo L = \tpo U D^{-1} D \tpo L$
Par unicité de la factorisation LU, il vient $D \tpo L = U$ et donc 
\[
    A = L D \tpo L
\]
    $\Rightarrow$ $A$ et $D$ ont la même signature.

    $A$ symétrique définie positive
    \begin{itemize}
        \item [$\Rightarrow$] Toutes les valeurs propres de A sont positives.
        \item [$\Rightarrow$] Toutes les valeurs propres de D sont positives

            $\Leftrightarrow u_{ii} > 0 \forall i$
    \end{itemize}

    Si $\delta_i = \sqrt{u_{ii}}$ et $\Delta = 
    \begin{pmatrix}
        \delta_1 & \dots & 0 \\
        & \ddots & \\
        0 & \dots & \delta_n
    \end{pmatrix}
    = diag(\sqrt{u_{ii}})$

    $D = \Delta^2 = \Delta \tpo \Delta$

    $\Rightarrow A = L \Delta \tpo \Delta \tpo L = (L\Delta)\tpo (L\Delta)$



\end{document}




