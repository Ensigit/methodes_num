% definit le type de document et ses options
\documentclass[a4paper,11pt]{article}

% des paquetages utiles classiques, en ajouter d'autres selon vos besoins
\usepackage[utf8]{inputenc}
\usepackage[T1]{fontenc}
\usepackage{amsmath}
\usepackage{amssymb,calc}
% \usepackage{fullpage}
% \usepackage{stmaryrd}
% \usepackage{url}
% \usepackage{xspace}
\usepackage[francais]{babel}

% Pour les matrices annotées ... 
\usepackage{blkarray}

% Pour les figures
\usepackage{graphicx}

% Pour les listes à puces
\usepackage{enumitem}

\usepackage{indentfirst}

% Évite un conflit entre french babel et enumitem
\frenchbsetup{StandardLists=true}

\usepackage[standard,framed]{ntheorem}
\usepackage{framed}

% Pour les matrices par blocs je crois ?
\makeatletter
\renewcommand*\env@matrix[1][*\c@MaxMatrixCols c]{%
  \hskip -\arraycolsep
  \let\@ifnextchar\new@ifnextchar
  \array{#1}}
\makeatother

% Pour des matrices agrandies :
\newenvironment{bigmatrix}[2]{%
  \renewcommand*{\arraystretch}{#1}% 
  \begin{pmatrix}[#2]
}{%
  \end{pmatrix}
}

% Pour des commentaires à droite d'équations
\newenvironment{rcases}
  {\left.\begin{aligned}}
  {\end{aligned}\right\rbrace}

\newenvironment{lcases}
  {\left\lbrace\begin{aligned}}
  {\end{aligned}\right.}


\newcommand{\reff}[1]{(\ref{#1})}


\setlength{\parindent}{30pt}
\setlength{\parskip}{1ex}
\setlength{\textwidth}{15cm}
\setlength{\textheight}{24cm}
\setlength{\oddsidemargin}{0.2cm}
\setlength{\evensidemargin}{-.7cm}
\setlength{\topmargin}{-.5in}

% des commandes pratiques pour ecrire des maths :
\newcommand{\dx}{\,dx}
\newcommand{\dt}{\,dt}
\newcommand{\ito}{,\dotsc,}
\newcommand{\R}{\mathbb{R}}
\newcommand{\N}{\mathbb{N}}
\newcommand{\C}{\mathbb{C}}
\newcommand{\Z}{\mathbb{Z}}
\newcommand{\Poly}[1]{\mathcal{P}_{#1}}
\newcommand{\abs}[1]{\left\lvert#1\right\rvert}
\newcommand{\norm}[1]{\left\lVert#1\right\rVert}
\newcommand{\pars}[1]{\left(#1\right)}
\newcommand{\bigpars}[1]{\bigl(#1\bigr)}
\newcommand{\set}[1]{\left\{#1\right\}}
\newcommand{\tpo}[1]{\,^t#1}


\DeclareMathOperator{\Sup}{Sup}
\DeclareMathOperator{\Max}{Max}

%
%\lstset{
%  literate=%
%  {À}{{\`A}}1 {Â}{{\^A}}1 {Ç}{{\c{C}}}1%
%  {à}{{\`a}}1 {â}{{\^a}}1 {ç}{{\c{c}}}1%
%  {É}{{\'E}}1 {È}{{\`E}}1 {Ê}{{\^E}}1 {Ë}{{\"E}}1% 
%  {é}{{\'e}}1 {è}{{\`e}}1 {ê}{{\^e}}1 {ë}{{\"e}}1%
%  {Ï}{{\"I}}1 {Î}{{\^I}}1 {Ô}{{\^O}}1%
%  {ï}{{\"i}}1 {î}{{\^i}}1 {ô}{{\^o}}1%
%  {Ù}{{\`U}}1 {Û}{{\^U}}1 {Ü}{{\"U}}1%
%  {ù}{{\`u}}1 {û}{{\^u}}1 {ü}{{\"u}}1%
%}

\newlength\dlf
\newcommand\alignedbox[2]{
  % #1 = before alignment
  % #2 = after alignment
    &
    \begingroup
    \settowidth\dlf{$\displaystyle #1$}
    \addtolength\dlf{\fboxsep+\fboxrule}
    \hspace{-\dlf}
    \boxed{#1 #2}
\endgroup
}

%%%%% PACKAGE AMSTHM :
% \newtheoremstyle{definition}
%   {20pt}   % ABOVESPACE
%   {20pt}   % BELOWSPACE
%   {}  % BODYFONT
%   {0pt}       % INDENT (empty value is the same as 0pt)
%   {\bfseries} % HEADFONT
%   {.}         % HEADPUNCT
%   {5pt plus 1pt minus 1pt} % HEADSPACE
%   {}          % CUSTOM-HEAD-SPEC

\newframedtheorem{ftheo}[theorem]{Theoreme}

\theoremstyle{plain} % default
\theorembodyfont{\normalfont}
\newframedtheorem{fdef}[definition]{Définition}

\renewtheorem{remark}{Remarque}

\newframedtheorem{coroll}{Corollaire}

\newtheorem{rappel}{Rappel}

\newtheorem{preuve}{Preuve}

\title{\huge \bfseries Méthode de Gauss pour les systèmes linéaires et factorisation LU}
\date{}


\begin{document}
\maketitle

Soit $A \in M_n(\R)$ inversible et $b \in \R^n$. La méthode de Gauss permet de résoudre le système $Ax=b$, $x \in \R^n$ en se ramenant à la résolution d'un système triangulaire.
Nous allons commencer par rappeler cette méthode classique de résolution des systèmes linéaires. Il s'agit d'une \underline{méthode directe}, càd qui donne la solution exacte un nombre fini d'opérations arithmétiques élémentaires.
Nous verrons ensuite que l'élimination de Gauss fournit une factorisation $A = LU$ (ou $PA = LU$, $P$ étant une matrice de permutation, dépendant du choix des pivots)
avec $L$ triangulaire inférieure et $U$ triangulaire supérieure.
Résoudre $Ax=b \Leftrightarrow PAx = Pb \Leftrightarrow LUx = Pb$ revient donc à :

\begin{enumerate}
    \item Factoriser $PA$
    \item Résoudre $Lc = Pb$ (étape de \underline{descente} $c_1 \to c_2 \to \dots \to c_n$
    \item Résoudre $Ux = c$ (étape de \underline{remontée} : $x_n \to x_{n-1} \to \dots \to x_1$
\end{enumerate}

Si on doit résoudre de nombreuses fois avec la même matrice :
\[
    Ax^{(k)}=b^{(k)}
\]

Schémas ``implicites'' pour des EDP, schémas itératifs pour des systèmes linéaires ou non linéaires, \dots) alors l'étape 1) qui est la plus coûteuse est effectuée une seule fois.


\section{Rappel de l'élimination de Gauss :}

Soit $A \in M_n(\R)$ inversible et $b \in \R^n$. On cherche $x \in \R^n$ tel que $Ax=b$, soit :

\begin{equation}
    \left\lbrace
    \begin{array}{ccc}
        a_{11}X_1 + a_{12}X_2+ \dots+ a_{1n}X_n & = & b_1 \\
        \vdots \\
        a_{n1}X_1 + a_{n2}X_2 + \dots + a_{nn}X_n & = & b_n
    \end{array}\right.
\end{equation}


En notant $L_i = (a_{i1}, \dots, a_{in})$ la $i^{ème}$ ligne de $A$, on a
\begin{equation}
    \left\lbrace
    \begin{array}{ccc}
        L_1 X = b_1 \\
        \vdots \\
        L_n X = b_n
    \end{array}\right.
    \tag{S}
    \label{eq:S}
\end{equation}


Si $a_{11} \neq 0$, on peut éliminer la variable $x_1$ dans les lignes 2 à $n$. On dit qu'on choisit $a_{11}$ comme \underline{pivot}.
$(S)$ équivaut à :
\[
    \left\lbrace
    \begin{array}{ccc}
        L_{1}X & = & b_1 \\
        (L_i - \frac{a_{i1}}{a_{11}}L_1) X & = & b_i - \frac{a_{i1}}{a_{11}}b_1 
    \end{array}\right.
    \hspace{2cm} i = 2..n
\]

Le nouveau système s'écrit $A^{(2)}X =b^{(2)}$ avec

\[
    A = 
    \begin{pmatrix}
        a_{11}^{(1)} & \cdots & a_{1n}^{(1)} \\
        0 \\
        \vdots \\
        0
    \end{pmatrix} 
    ,
    (a_{ij}^{(1)} = a_{ij})
\]

Ligne $i = L_i - l_{i1}L_1$ avec $l_{i1} = \frac{a_{i1}}{a_{11}}$

$b_i^{(2)} =  b_i - l_{i1}b_1$

Si $a_{11}$ on permute la 1ère ligne de \reff{eq:S} avec une autre ou $a_{i1} \neq 0$. Cela est toujours possible puisque $A$ est inversible.
On effectue la même procédure que précédemment expliqué.

Le système $A^{(2)} X = b^{(2)}$ contient un sous-système de dimension $n-1$ pour $x_2\dots x_n$
On répète la même procédure sur le sous-système pour éliminer $X_2$ des lignes 3 à $n$.

On continue ainsi et on déduit
\[
    A^{(3)}X = b^{(3)}, A^{(4)}X = b^{(4)}, \dots, A^{(n)}X=b^{(n)}
\]
Le dernier système obtenu est triangulaire. Notons $A^{(n)}=U$, $b^{(n)}=C$.

\begin{equation}
    \left\lbrace
    \begin{array}{ccccccc}
        u_{11}x_1 & + & \dots & + & u_{1n}x_n & = & c_1 \\
                  & u_{22}x_2 + & \dots & + & u_{2n}x_n & = & c_2 \\
        & & & \ddots \\
        & & & & u_{nn}x_n & = & c_n \\
    \end{array}\right.
    \tag{S'}
    \label{eq:S2}
\end{equation}

\reff{eq:S2} est facile à résoudre : \underline{``étape de remontée''}.
\[
    x_{nn} = \frac{c_n}{x_{nn}}, x_i = \frac{1}{u_{ii}}(c_i - \sum_{j=i+1}^{n}u_{ij}x_j )
\] pour $i = n-1, \dots, 1$

\begin{remark}
    On appelle \underline{``factorisation''} le calcul de $U$.
\end{remark}









\section{Factorisation LU}
Nous avons vu que l'élimination de Gauss peut conduire à permuter des lignes de A puisqu'on a besoin de ``pivots'' non nuls $a_{11}^{(1)}, a_{22}^{(2)}$ etc \dots Les cas où ces pivots sont voisins de 0 conduisent à des problèmes numériques (voir plus loin). Il est donc fréquent d'effectuer des permutations des lignes de A lors de l'élimination de Gauss.
Nous allons tout d'abord voir que ces permutations sont une traduction matricielle simple.

Notons ${p_1,p_2,\dots,p_n}$ une permutation des entiers ${1,2,\dots,n}$ et $(e_1,\dots,e_n)$ la base canonique de $\R^n$. On appelle \underline{matrice de permutation} une matrice de la forme :

\[
    P =
    \begin{pmatrix}[c|c|c|c]
        e_{p_1} & e_{p_2} & \dots & e_{p_n}\\
\end{pmatrix} 
\]

On a $p_{l_i} = e_{p_i}$ et :

\[
    P
    \begin{pmatrix}
        x_1 \\ x_2 \\ \vdots \\ x_n
    \end{pmatrix}
    =
    \begin{pmatrix}
        \vdots \\ \vdots \\ x_j \\ \vdots
    \end{pmatrix}
    \text{$\leftarrow$ ligne $p_j$}, \hspace{1cm}
    P
    \underbrace{\begin{pmatrix}
        L_1 \\
        L_2 \\
        \vdots \\
        L_n
    \end{pmatrix}
    }_\text{A}
    =
    \begin{pmatrix}
        \vdots \\
        \vdots \\
        L_j \\ 
        \vdots
    \end{pmatrix}
    \text{$\leftarrow$ ligne $p_j$}
\]

Soit $A \in M_n(\R)$. Notons $P$ la matrice correspondant aux permutations effectuées sur les lignes de $A$ dans l'algorithme du paragraphe \textbf{1}.
On a donc $PA = \tilde{A}$, où l'élimination de Gauss sur $\tilde{A}$ se fait sans permutation. Le passage de $A^{(j-1)}$ à $A^{(j)}$ s'écrit (même notations qu'auparavant) : 
\[
    \text{Ligne $i$} = L_i - L_j \times \left( \frac{a^{(j-1)}_{ij}}{a^{(j-1)}_{jj}} \right), \hspace{0.5cm} i = j+1, \dots , n
\]

Soit matriciellement :
\[
    A^{(j)} = T_j A^{(j-1)}, \hspace{0.5cm} T_j =
    \begin{pmatrix}[c|c]
        \begin{matrix}[ccc]
        & & \\
        & I_j & \\
        & & \\
        \end{matrix} 
        &
        \begin{matrix}
            \scalebox{1.5}{0}
        \end{matrix}
        \\ \hline
        \begin{matrix}[ccc|c]
            & & & -l_{j+1,j} \\
            & \scalebox{1.5}{0} & & \vdots \\
            & & & -l_{n,j}
        \end{matrix}
        &
        \begin{matrix}[ccc]
        & & \\
        & I_{n_j} & \\
        & & \\
        \end{matrix} 
    \end{pmatrix}
\]

\[
    l_{ij} = \frac{a^{(j-1)}_{ij}}{a^{(j-1)}_{jj}} , \hspace{0.5cm} I_k = \text{matrice identité de taille $k$}
\]

    En effet : 
\[
    \begin{blockarray}{cccc}
        \begin{block}{c(ccc)}
            & & 0 & \\
            \cline{2-4}
            & & 0 & \\
            \cline{2-4}
            \text{\scalebox{0.7}{ligne $i \rightarrow$}} & 0 \dots 0 & 1 & 0 \dots 0 \\
            \cline{2-4}
            & & 0 & \\
            \cline{2-4}
            & & 0 & \\
        \end{block}
        & & \text{\scalebox{0.7}{$\uparrow$ colonne $j$}} & & \\
    \end{blockarray}
    \times
    \begin{pmatrix}
        L_1 \\ \hline
        L_2 \\ \hline
        \vdots \\ \hline
        L_n
    \end{pmatrix}
    =
    \begin{pmatrix}
        0 \\ \hline
        0 \\ \hline
        L_j \\ \hline
        0 \\ \hline
        0
    \end{pmatrix} \text{$\leftarrow$ ligne $j$}
\]

On a donc :
\[
    U = A^{(n-1)} = T_{n-1} \times T_{n-2} \times \dots \times T_1 \times PA = TPA
\]
Avec $U$ triangulaire supérieure et $T$ triangulaire inférieure (produit de matrices triangulaires inférieures).

Donc on a $PA=LU$ avec $L=T^{-1}$ triangulaire inférieure (l'inverse d'une matrice triangulaire inférieure l'est aussi).

\begin{remark}
    Dans $\ref{eq:S2}$ on a $C=TPb$ et donc $LC=Pb$.
\end{remark}

% l_ij à afficher en plus grand ... 
Soit maintenant :\[
    \tilde{L} =
    \begin{pmatrix}
        1 & & \scalebox{1.2}{0} \\
        & \ddots & \\
        l_{ij} & & 1
    \end{pmatrix}
\]

On a : 
\[
    T_1 \tilde{L} =
    \begin{pmatrix}
        1      &        &        & 0 \\
        0      & 1      &        &   \\
        \vdots &        & \ddots &   \\
        0      & l_{ij} &        & 1
    \end{pmatrix}
    , \dots , T_{n-1} \times T_{n-2} \times T_1 \tilde{L} = I
\]

Donc $L = \tilde{L}$. Nous avons donc montré le résultat suivant :

\begin{ftheo}[Factorisation LU d'une matrice inversible]
    Soit $A \in M_n(\R)$ inversible. Il existe une matrice de permutation $P$
et deux matrices triangulaires L (triangulaire inférieure de diagonale unité) et $U$ (triangulaire supérieure inversible) telles que $PA = LU$.

    Cette décomposition est donnée explicitement par l'élimination de Gauss, avec 
    (les coefficients $l_{ij}$ sont ceux de l'élimination de Gauss sur $\tilde{A}$) :

    \[
        L = \begin{pmatrix}
            1 & & 0 \\
            & \ddots & \\
            l_{ij} & & 1 \\
        \end{pmatrix}
        , U =
        \begin{pmatrix}
            \ddots & & U_{ij} \\
            & \ddots & & \\
            0 & &
        \end{pmatrix}
    \]

    Cette factorisation est unique lorsque l'on fixe P.
    \label{th:factoLU}
\end{ftheo}

\begin{remark}
    L'unicité s'obtient simplement :
    $PA$ est inversible (puisque $P$ et $A$ le sont).

    Si $PA = L_1 U_1 = L_2 U_2$ alors $L_i$ et $U_i$ sont inversibles et donc
    $L^{-1}_2 L_1 = U_2 U^{-1}_1$. Le membre de droite est triangulaire supérieur, celui de gauche et triangulaire inférieur de diagonale unité.

    Donc $L^{-1}_2 L_1 = U_2 U^{-1}_1 = I$, i.e. $L1 = L2$ et $U_1 = U_2$.
\end{remark}

\vspace{0.7cm}
Les permutations effectuées lors de l'élimination de Gauss sont très importantes d'un point de vue numérique (voir plus loin).
Bien sûr, si l'on raisonne en arithmétique exacte (sans tenir compte des erreurs d'arrondi)
on voit dans l'algorithme de Gauss que les cas nécessiteant une permutation sont
exceptionnels (cela se produit lorsque $a_{11}$ ou $a^{(i)}_{i+1,i+1}=0$ pour certaines valeurs de $i$).

On a plus précisément le résultat suivant :

\begin{ftheo}
    Dans le théorème \ref{th:factoLU}, si les $n$ sous-matrices
    \[\Delta = \begin{pmatrix}
        a_{11} & \dots a_{1i} \\
        \vdots & & \vdots \\
        a_{i1} & \dots & a_{ii}
    \end{pmatrix}
, (1 \leq i \leq n) \]
    sont inversibles, alors on peut fixer $P = I$.
\end{ftheo}

\begin{preuve}
    $a_{11} \ne 0$ donc la 1\up{ère} de l'élimination de Gauss ne nécessite pas de permutation.
    Supposons qu'on ait $j$ étapes sans permutation :
    \[
        A^{(j)} =
        \begin{pmatrix}[c|c]
            U^{(j)} & X \\ \hline
            0 & X
        \end{pmatrix}
        = T_j T_{j-1} \times \dots \times T_1 \times A =
        \begin{pmatrix}[c|c]
            \begin{matrix}
                1 & & 0 \\
                & \ddots & \\
                X & & 1
            \end{matrix}
            & \begin{matrix}[ccc]\\ & 0 & \\ \end{matrix}
            \\ \hline
            \begin{matrix}[ccc]\\ & X & \\ \end{matrix}
            & \begin{matrix}[ccc]\\ & I & \\ \end{matrix}
        \end{pmatrix}
        \begin{bigmatrix}{2.5}{c|c}
            \Delta_j & X \\
            \hline
            X & X
        \end{bigmatrix}
    \]
\end{preuve}

\section{Le coût de la méthode de Gauss}

\begin{eqnarray}
    \begin{split}
        Ax = b & \Leftrightarrow & PAX = Pb \\
        & \Leftrightarrow & LUX = Pb
    \end{split}
    \label{S1}
\end{eqnarray}

Résoudre \ref{S1} en 3 étapes :
\begin{enumerate}
    \item Factoriser $A$ ($PA = LU$)
    \item Résoudre $LC = Pb$ (méthode de remontée)
    \item Résoudre $UX = c$ (méthode de descente)
\end{enumerate}

\vspace{1cm}
\begin{remark}
    \begin{enumerate}
        \item L'étape 1. est la plus coûteuse
        \item Utiliser Fact-LU quand on a plusieurs systèmes linéiaires. Avec la même matrice à résoudre : $AX^{(i)}=b^{(i)}, \; i = 1,2 \dots$
    \end{enumerate}
\end{remark}

\vspace{1cm}

* Factorisation $A=LU \; \; \; (P=I)$, à l'étape 1 de la factorisation : passage $A \to A^{(2)}$

\begin{enumerate}
    \item $(n-1)$ divisions (calcul de $l_{21}, \dots, l_{n1}$
    \item $2n(n-1)$ multitplications et additions (calcul $a_{ij}-l_{i1}a_{1j}, 2 \leq i \leq n, 1 \leq j \leq n$
    \item Passage $b$ à $b^{(2)}$, $2(n-1)$ multiplications et additions ($b_i - l_{i1}b_{1}, 2 \leq i \leq n$
\end{enumerate}

Il faut renouveler cette procédure pour les sous-systèmes de taille $n-1,n-2,\dots,2$

\underline{Au total}

\begin{equation*}
    \begin{rcases}
        \sim 2 \sum_{i=1}^n (n-i)^2 = 2.\frac{1}{3}n (n-\frac{1}{2})(n-1) \hspace{2cm}& +, *\\
        \sim 3 \sum_{i=1}^n (n-i) = 3.\frac{1}{2}n(n-1) \hspace{2cm} & +,*,\div
    \end{rcases}
    \text{$\Rightarrow \frac{2}{3}n^3+O(n^2)$}
\end{equation*}

* Réoslution d'un système triangulaire (étape de remontée)
(schéma pas pris)
\[
    X_n = \frac{b_n}{U_{nn}}
    x_i = \frac{1}{U_{2i}}(b_i - \sum_{k=i+1}^n a_{iK}X_K
\]

\begin{equation*}
    \text{$O(n^2)$}
    \begin{lcases}
        \text{n divisions}
        1 + 2 + 3 + \dots + n - 1 & = & \sum_{i=1}^{n-1}i = \frac{1}{2}(n-1)(n-2)\\
        1 + 2 + 3 + \dots + n - 1 & = & \sum_{i=1}^{n-1}i = \frac{1}{2}(n-1)(n-2)
    \end{lcases}
\end{equation*}

$\Rightarrow$ Donc la méthode de Gauss nécessite $\frac{2}{3}n^3+O(n^2)$ opérations.

\vspace{0.5cm}
\begin{remark}
    \begin{enumerate}
        \item Le pivot partiel a un coût en $O(n^2)$.
            \begin{itemize}
                \item [$\rightarrow$] Trouver le max parmi $n,n-1, \dots, 2, 1$ coeff.
            \end{itemize}

        \item Le pivot toal a un coût en $O(n^3)$
            \begin{itemize}
                \item [$\rightarrow$] Trouver le max parmi $n^2, (n-1)^2, \dots , 2^2, 1^2$ coeff.
            \end{itemize}
    \end{enumerate}
\end{remark}

\[
   \begin{pmatrix}[cc|c]
  1 & 2 & 3\\
  \hline
  4 \; \; \vline & 5 & 9 \\
  1 & 2 & 3
\end{pmatrix} 
\]
\end{document}




