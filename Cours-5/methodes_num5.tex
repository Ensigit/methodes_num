% definit le type de document et ses options
\documentclass[a4paper,11pt]{article}

% des paquetages utiles classiques, en ajouter d'autres selon vos besoins
\usepackage[utf8]{inputenc}
\usepackage[T1]{fontenc}
\usepackage{amsmath}
\usepackage{amssymb,calc}
% \usepackage{fullpage}
% \usepackage{stmaryrd}
% \usepackage{url}
% \usepackage{xspace}
\usepackage[francais]{babel}

% Pour les matrices annotées ... 
\usepackage{blkarray}

% Pour les figures
\usepackage{graphicx}

% Pour les listes à puces
% \usepackage{enumitem}

\usepackage{enumerate}

\usepackage{indentfirst}

% Évite un conflit entre french babel et enumitem
\frenchbsetup{StandardLists=true}

\usepackage[standard,framed]{ntheorem}
\usepackage{framed}

% Pour les matrices par blocs je crois ?
\makeatletter
\renewcommand*\env@matrix[1][*\c@MaxMatrixCols c]{%
  \hskip -\arraycolsep
  \let\@ifnextchar\new@ifnextchar
  \array{#1}}
\makeatother

% Pour annoter les matrices :
\usepackage{kbordermatrix} 
\renewcommand{\kbldelim}{(} % change default array delimiters to parentheses
\renewcommand{\kbrdelim}{)}


% Pour des matrices agrandies :
\newenvironment{bigmatrix}[2]{%
  \renewcommand*{\arraystretch}{#1}% 
  \begin{pmatrix}[#2]
}{%
  \end{pmatrix}
}

% Pour des commentaires à droite d'équations
\newenvironment{rcases}
  {\left.\begin{aligned}}
  {\end{aligned}\right\rbrace}

\newenvironment{lcases}
  {\left\lbrace\begin{aligned}}
  {\end{aligned}\right.}


\newcommand{\reff}[1]{(\ref{#1})}


\setlength{\parindent}{30pt}
\setlength{\parskip}{1ex}
\setlength{\textwidth}{15cm}
\setlength{\textheight}{24cm}
\setlength{\oddsidemargin}{0.2cm}
\setlength{\evensidemargin}{-.7cm}
\setlength{\topmargin}{-.5in}

% des commandes pratiques pour ecrire des maths :
\newcommand{\dx}{\,dx}
\newcommand{\dt}{\,dt}
\newcommand{\ito}{,\dotsc,}
\newcommand{\R}{\mathbb{R}}
\newcommand{\N}{\mathbb{N}}
\newcommand{\C}{\mathbb{C}}
\newcommand{\Z}{\mathbb{Z}}
\newcommand{\Poly}[1]{\mathcal{P}_{#1}}
\newcommand{\abs}[1]{\left\lvert#1\right\rvert}
\newcommand{\norm}[1]{\left\lVert#1\right\rVert}
\newcommand{\pars}[1]{\left(#1\right)}
\newcommand{\bigpars}[1]{\bigl(#1\bigr)}
\newcommand{\set}[1]{\left\{#1\right\}}
\newcommand{\tpo}[1]{\,^t#1}


\DeclareMathOperator{\Sup}{Sup}
\DeclareMathOperator{\Max}{Max}
\DeclareMathOperator{\Det}{Det}
\DeclareMathOperator{\diag}{diag}
\DeclareMathOperator{\Min}{Min}

%
%\lstset{
%  literate=%
%  {À}{{\`A}}1 {Â}{{\^A}}1 {Ç}{{\c{C}}}1%
%  {à}{{\`a}}1 {â}{{\^a}}1 {ç}{{\c{c}}}1%
%  {É}{{\'E}}1 {È}{{\`E}}1 {Ê}{{\^E}}1 {Ë}{{\"E}}1% 
%  {é}{{\'e}}1 {è}{{\`e}}1 {ê}{{\^e}}1 {ë}{{\"e}}1%
%  {Ï}{{\"I}}1 {Î}{{\^I}}1 {Ô}{{\^O}}1%
%  {ï}{{\"i}}1 {î}{{\^i}}1 {ô}{{\^o}}1%
%  {Ù}{{\`U}}1 {Û}{{\^U}}1 {Ü}{{\"U}}1%
%  {ù}{{\`u}}1 {û}{{\^u}}1 {ü}{{\"u}}1%
%}

\newlength\dlf
\newcommand\alignedbox[2]{
  % #1 = before alignment
  % #2 = after alignment
    &
    \begingroup
    \settowidth\dlf{$\displaystyle #1$}
    \addtolength\dlf{\fboxsep+\fboxrule}
    \hspace{-\dlf}
    \boxed{#1 #2}
\endgroup
}

%%%%% PACKAGE AMSTHM :
% \newtheoremstyle{definition}
%   {20pt}   % ABOVESPACE
%   {20pt}   % BELOWSPACE
%   {}  % BODYFONT
%   {0pt}       % INDENT (empty value is the same as 0pt)
%   {\bfseries} % HEADFONT
%   {.}         % HEADPUNCT
%   {5pt plus 1pt minus 1pt} % HEADSPACE
%   {}          % CUSTOM-HEAD-SPEC

\newframedtheorem{ftheo}[theorem]{Theoreme}

\theoremstyle{plain} % default
\theorembodyfont{\normalfont}
\newframedtheorem{fdef}[definition]{Définition}

\renewtheorem{remark}{Remarque}

\newframedtheorem{coroll}{Corollaire}

\newtheorem{rappel}{Rappel}

\newtheorem{preuve}{Preuve}

\title{\huge \bfseries Résolution numérique d'équations non linéaires}
\date{}


\begin{document}
\maketitle

De nombreux problèmes issus notamment de la physique conduisent à la résolution d'équations non linéaires,
\[
    f(x)=0, \; f \in \mathcal{C}^1(\R^n,\R^n)
\]

\section{Méthode des approximations successives}

À partir d'une équation $f(x) = 0$ ($f : \R^n \to R^n$ de classe $\mathcal{C}^1$),
on peut se ramener à un problème de point fixe :
\begin{equation}
    x = \Phi(x)
    \label{eq:ptfixe}
\end{equation}
avec $\Phi : \R^n \to \R^n$ de classe $\mathcal{C}^1$. On peut poser par exemple :
\begin{equation*}
    \Phi(x) = x - B f(x)
\end{equation*}
avec $B \in M_n(\R)$ inversible.

Pour résoudre (\ref{eq:ptfixe}), on se donne une condition initiale $x_0 \in \R^n$ (la plus proche possible d'une solution de (\ref{eq:ptfixe})) et on considère la méthode itérative :
\begin{equation}
    x_{k+1} = \Phi(x_k)
    \label{eq:methodeiterative}
\end{equation}

Nous allons étudier la convergence de ce type de méthodes itératives.

\begin{fdef}
    Soit $a$ un point fixe de $\Phi$ ($\Phi(a) = a$).
    \begin{enumerate}[i)]
        \item $a$ est stable au sens de Lyapunov si
            \[
                \forall \varepsilon > 0, \exists \eta \; / \; \norm{x_0 - a} < \eta \implies \norm{x_k - a} < \varepsilon \hspace{1cm} \forall k \geq 0
            \]

        \item $a$ est instable s'il n'est pas stable au sens de Lyapunov.

        \item $a$ est asymptotiquement stable s'il est stable au sens de Lyapunov et
            \[
                \exists r \; / \; \norm{x_0 - a} < r \implies x_k \xrightarrow{k \to +\infty} a
            \]
    \end{enumerate}
\end{fdef}

Lorsque $a$ est asymptotiquement stable, la méthode (\ref{eq:methodeiterative}) permet de calculer numériquement $a$ à partir d'une condition initiale $x_0$ ``suffisamment proche'' de $a$.

\begin{ftheo}
    Soit $\Omega$ un ouvert de $\R^n$ et $\Phi : \Omega \to \R^n$ de classe $\mathcal{C}^1$.
    Soit $a \in \Omega$ un point fixe de $\Phi$, i.e. $\Phi(a) = a$. Alors :
    \begin{enumerate}[a)]
        \item Si $\rho \Big(D\Phi(a) \Big) < 1$ alors $a$ est asymptotiquement stable.
        \item Si $\rho \Big(D\Phi(a) \Big) > 1$ alors $a$ est instable.
    \end{enumerate}
\end{ftheo}

\begin{rappel*}
    $D\Phi(a) \in M_n(\R)$ est définie par :
    \[
        \left.
        \parbox{4cm}{
            \[
                D \Phi (a) = \left( \frac{\partial \Phi_i}{\partial x_j}(a) \right)_{1 \leq i,j \leq n}
            \]
        } \hspace{1cm} \right \rbrace
        \begin{tabular}{@{}c@{}}
            différentielle de $\Phi$ au point $a$, \\ matrice Jacobienne de $\Phi$ au point $a$ 
        \end{tabular}
     \]
\end{rappel*}

\begin{preuve}[du a)]
    Notons $x_k = a + e_k$.
    \begin{equation*}
        \left\lbrace
        \begin{array}{ccc}
            x_{k+1} = \Phi (x_k) & \implies & e_{k+1} = \Phi(a+e_k) - \Phi(a) \\
            a = \Phi(a)
        \end{array}
        \right.
    \end{equation*}

    On utilise un développement de Taylor à l'ordre 1 :
    \[
        \Phi(a+e_k) = \Phi(a) + D\Phi(a) e_k + \norm{e_k}\varepsilon(e_k)
    \]
    avec $\norm{\varepsilon(e_k)} \to 0$ quand $e_k \to 0$.


    \vspace{0.5cm}
    Donc $e_{k+1} = D\Phi(a) e_k + o(\norm{e_k})$.

    Si $\rho \big(D\Phi(a) \big) < 1$, il exsite une norme matricielle induite pour
laquelle $\norm{D\Phi(a)}<1$.

    Donc $\exists \eta > 0$ et $\alpha < 1$ tels que si $\norm{e_k} < \eta$ :
    \[
        \norm{e_{k+1}} \leq \alpha \norm{e_k}
    \]

    Donc si $\norm{e_0} < \eta$, $\norm{e_k} \leq \alpha^k \norm{e_0} \xrightarrow{k \to +\infty}0$
\end{preuve}

\begin{remark}
    \begin{enumerate}[-]
        \item Ce résultat donne la convergence \underline{locale} de la méthode :
              convergence de  $(x_k)$ vers un point fixe $a$ de $\Phi$ si
              $\rho \big(D\Phi(a) \big)<1$ et $\norm{x_0-a}$ assez petit.
        \item La solution de (\ref{eq:ptfixe}) n'est pas forcément unique.
        \item a) $\implies \norm{x_{k+1}-a} \leq \alpha \norm{x_k-a}$ avec
              $\alpha < 1$, et plus $\rho \big( D\Phi(a) \big)$ est petit, plus
              $\alpha$ est petit. On dit que la convergence est (au moins)
              \underline{linéaire}.
        \item Sous l'effet des termes non linéaires, dans certains cas la méthode
            numérique (\ref{eq:methodeiterative}) peut être localement convergente
            avec $\rho \big( D\Phi(a) \big) = 1$. \underline{Exemple :} $x_{k+1} = x_k - x_k^3$, point fixe 0 asymptotiquement stable.
    \end{enumerate}
\end{remark}

\subsection*{Critères d'arrêt :}
On se donne une tolérance absolue $tol$ (on pourrait aussi travailler en relatif)
\begin{enumerate}[a)]
    \item
        \[
              \norm{x_k - x_{k-1}} < tol
        \]
        Cela indique également que $\norm{\Phi(x_{k-1} - x_{k-1}} < tol$,
        c'est-à-dire que $x_{k-1}$ est ``presque'' solution de $\Phi(x) = x$.
        
    \item Lorsque $\Phi$ est une contraction sur un sous-ensemble fermé $E$ de
          de $\R^n$, on sait que $\Phi$ admet un unique point fixe $a$ dans $E$.
          Si $\alpha \in ]0,1[$ désigne le facteur de contraction de $\Phi$ on
          montre que si $x_{k-1} \in E$ alors $\norm{x_k-a} \leq \frac{\alpha}{1-\alpha} \norm{x_k - x_{k-1}}$.
          
          Fixer le critère d'arrêt $\norm{x_k - x_{k-1}} < tol \times \big(\frac{1}{\alpha} - 1 \big)$ et $x_{k-1} \in E$ garantit que $\norm{x_k - a} < tol$.


    \item Un critère intéressant peut être obtenu lorsque :
        \[
        \frac{\norm{x_k - x_{k-1}}}{\norm{x_{k-1}-x_{k-2}}} \xrightarrow{k\to+\infty} \lambda \in ]0,1[ 
        \]

        Cette propriété est vérifiée avec $\lambda = \rho \big(D\Phi(a) \big)$ et pour
        presque toute condition initiale $x_0 \approx 0$ si $\lambda$ \underline{ou} $-\lambda$
        est une valeur propre réelle simple de $D\Phi(a)$, avec toutes les autres
        valeurs propres de module $< \lambda$.

        (alors $x_k = a + V.(\pm \lambda)^k + o(\lambda^k$), $V$ vecteur propre
        associé à $\pm \lambda$)

        Alors pour $k$ assez grand et $p \geq k$

        \begin{align*}
            & \norm{x_k - x_p} \leq \norm{x_k - x_{k+1}} + \norm{x_{k+1} - x_{k+2}}
            + \dots + \norm{x_{p-1}-x_p} \\ 
            \\
            \implies & \norm{x_k - a} \leq \sum_{j \geq k} \norm{x_j - x_{j+1}} \hspace{1cm} \text{(on fait tendre $p$ vers $+\infty$)}
        \end{align*}
        
        On fait maintenant l'approximation :
        \begin{align*}
            \sum_{j \geq k} \norm{x_j - x_{j+1}} & \simeq \norm{x_k - x_{k+1}} \times \sum_{j \geq 0} \lambda ^j \simeq \frac{\lambda}{1 - \lambda} \norm{x_k - x_{k-1}} \\
            & \simeq \frac{\norm{x_k - x_{k-1}}}{\norm{x_{k-1}x_{k-2}}} \times \frac{1}{1-\frac{\norm{x_k - x_{k-1}}}{\norm{x_{k-1}-x_{k-2}}}} \norm{x_k - x_{k-1}}
        \end{align*}
\end{enumerate}
\end{document}




