\chapter{Approximation de problèmes aux limites par différences finies}

\section{Introduction}

On considère ici des problèmes aux limites en une dimension,
linéaires, du second ordre :

\begin{subnumcases}{}
- u''(x) + p(x) u'(x) + q(x) u(x) = f(x) \hspace{1cm} x \in ]a,b[ \label{eq:1-pbvp-1}\\
    \left.\begin{array}{c}
        u(a) = \alpha \\
        u(b) = \beta
    \end{array} \right\rbrace \label{eq:1-pbvp-cl}
\end{subnumcases}
où $p,q,f \in \Co^0([a,b])$ et $\alpha,\beta \in R$. On a une équation
différentielle linéaire du 2\up{nd} ordre, à coefficients a priori variables,
augmentée des conditions aux limites \eqref{eq:1-pbvp-cl}.
Ces conditions fixent la valeur de $u$ au bord du domaine : on parle de
\textbf{conditions de Dirichlet}.

Lorsque les conditions aux limites fixent la valeur de $u'$ au bord 
($u'(a)=\alpha,u'(b)=\beta$), on parle de conditions de \textbf{Neumann}.

\subsection*{Exemples issus de la physique}
\begin{enumerate}[label=\alph*)]
    \item 
        \[
            \left\lbrace
            \begin{array}{lll}
                \dfrac{d^2\phi}{dr^2} + \dfrac{1}{r} \dfrac{d\phi}{dr} = \lambda \phi & , & r \in ]r_a,R[
                    \\ [5pt]
            \phi'(r_a) = \alpha
            \\ [2pt]
            \phi'(R) = 0
            \end{array}
            \right.
        \]

        Equations de Debye-Hückel donnant le potentiel électrique $\phi$ autour
        d'un cylindre chargé (de rayon $r_a$) polongé dans une solution
        ionique ($r_a < r < R$).

    \item 
        \[
            \left\lbrace
            \begin{array}{lll}
            - \dfrac{d}{dx}\left( k(x) \dfrac{du}{dx} \right) = f(x) & , & x \in ]0,L[
            \\ [5pt]
            u(0) = 0
            \\ [2pt]
            u(L) = 0 
            \end{array}
            \right.
        \]

        Equation de la chaleur stationnaire donnant la température $u$ dans
        une barre de longueur $L$ chauffée et de conductivité $k(x)$ variable.
        ($f(x) = $ quantité de chaleur fournie par unité de temps et de longueur).
        La température aux extrémités de la barre est maintenue à 0.
\end{enumerate}

Lorsque $q \geq 0$, le résultat suivant assure que \eqref{eq:1-pbvp-1} - \eqref{eq:1-pbvp-cl}
possède une solution unique. Mais on n'a pas en général d'expression explicite
de $u$ quand $p$ ou $q$ dépendent de $x$.

\begin{ftheo}
    Si $q(x) \geq 0$ sur $[a,b]$ alors \eqref{eq:1-pbvp-1} - \eqref{eq:1-pbvp-cl}
    \textbf{admet une unique solution} $u \in \Co^2([a,b])$.
    
    De plus, si $p,q,f \in \Co^k([a,b])$
    alors $\boxed{u \in \Co^{k+2}([a,b])}$.
\end{ftheo}

\begin{preuve}[de l'unicité]
    Si $u_1$ et $u_2$ vérifient \eqref{eq:1-pbvp-1} - \eqref{eq:1-pbvp-cl}, alors
    $v = u_1 - u_2$ est solution du problème homogène :
    \[
        \left\lbrace
        \begin{array}{l}
            -v'' + pv' + qv = 0 \\
            v(a) = v(b) = 0
        \end{array}
        \right.
    \]

    Notons $r(x) = e^{-\int p \dx}$ et $s(x) = q(x) e^{-\int p \dx}$.
    En multipliant l'équation par $r$ on a :
    \[
       -\left( r(x) v'(x) \right)' + s(x) v(x) = 0
    \]

    On multiplie maintenant cette équation par $v$ et on intègre sur $[a,b]$,
    en utilisant les conditions aux limites :

    \[
        \int_a^b r (x) v'^2 \dx + \int_a^b s(x) v^2 \dx = 0
    \]
    avec $r>0$ et $s \geq 0$

    Donc $\int_a^b r(x) v'^2 \dx = 0 \implies v'=0$ (puisque $r>0$) $\implies v = 0$ à cause des conditions aux limites.
\end{preuve}

L'existence d'une solution $u \in \Co^2([0,1])$ peut être obtenue de différentes façons
(méthode de variation de la constante, méthodes d'analyse fonctionnelle,
par exemple formulation variationnelle).
La régularité $\Co^k$ de $u$ s'obtient simplement par récurrence sur $k$.

\section{Discrétisation par différences finies}

Nous allons étudier comment calculer $u$ numériquement.

\[
    (p) \;
    \left\lbrace
    \begin{array}{cc|c}
        -u'' + p(x) u' + q(x) u = f(x) &  & x \in ]a,b[ \\
            u(a) = \alpha &  & p,q,f \in \Co^0([a,b]) \\
            u(b) = \beta &  & q \geq 0 \text{ sur } [a,b]
    \end{array}
    \right.
\]

On discrétise $[a,b]$ suivant les points $x_i = a + ih$ ($i=0,\dots,N+1$)
avec $h = \frac{b-a}{N+1}$. On notera $u_i$ la valeur approchée de
$u(x_i)$ à calculer et $U = \tpo (u_1, \dots, u_N)$, $(u_0 = \alpha, u_{N+1} = \beta)$.

On approche $u''(x_i)$ et $u'(x_i)$ en utilisant un développement de Taylor
de $u$ en $x_i$.

Si $u \in \Co^4([a,b])$ :
\[
    u(x_{i+1}) = u(x_i + h) = u(x_i) + h \: u'(x_i) + \frac{h^2}{2} u''(x_i)
    + \frac{h^3}{6} u^{(3)}(x_i) + \frac{h^4}{24} u^{(4)}(\theta_i^+)
\]
\hfill avec $\theta_i^+ \in ]x_i,x_{i+1}[$

\[
    u(x_{i-1}) = u(x_i - h) = u(x_i) - h \: u'(x_i) + \frac{h^2}{2} u''(x_i)
    - \frac{h^3}{6} u^{(3)}(x_i) + \frac{h^4}{24} u^{(4)}(\theta_i^-)
\]
\hfill avec $\theta_i^- \in ]x_{i-1},x_i[$

    Donc :
    \[
        \frac{u(x_{i+1}) - u(x_{i-1})}{2h} =  u'(x_i) + \mathcal{O}(h^2)
    \]
    \[
        \frac{u(x_{i+1}) - u(x_i) + u(x_{i-1})}{h^2} =  u''(x_i) + \mathcal{O}(h^2)
    \]

    Notons $p_i = p(x_i)$, $q_i = q(x_i)$, $f_i = f(x_i)$. On a donc :
    \[
        \frac{2u(x_i) - u(x_{i+1}) - u(x_{i-1})}{h^2} + p_i \frac{u(x_{i+1}) - u(x_{i-1})}{2h} + q_i u(x_i) - f_i = e_i
    \]
    \hfill pour tout $i=1,\dots,N$, avec $e_i = \mathcal{O}(h^2)$

    L'approximation du problème $(p)$ par différences finies consiste à
    résoudre :

    \[
        (S) \;
        \left\lbrace
        \begin{array}{ccc}
            \dfrac{2u_i - u_{i+1} - u_{i-1}}{h^2} + p_i \dfrac{u_{i+1} - u_{i-1}}{2h} + q_i u_i - f_i = 0 & & \hspace{0.5cm} i=1,\dots,N \\ [15pt]
            u_0 = \alpha, \; u_{N+1} = \beta & &
        \end{array}
        \right.
    \]

    La solution de $(p)$ vérifie donc $(S)$ à une erreur $e_i$ près qui est
    $\mathcal{O}(h^2)$ lorsque $h \longrightarrow 0$. On dit que le schéma
    $(S)$ est \textbf{consistant}.

    On appelle $e_i$ l'erreur de troncature (ou erreur de consistance) du schéma
    $(S)$ au point $x_i$.

    Celle-ci étant $\mathcal{O}(h^2)$ (pour $u$ assez régulière) on dit que le schéma
    $(S)$ est \textbf{d'ordre 2}. Le schéma serait d'ordre $p$ avec une erreur
    de troncature en $\mathcal{O}(h^p)$.

    \begin{remark}
        On a ici 
        \[
            \MaxI{i\leq i \leq N} |e_i| \leq \dfrac{h^2}{6} \left( \frac{1}{2}
            \norm{u^{(4)}}_\infty + \norm{p}_\infty \norm{u^{(3)}}_\infty \right)
        \]

        Le problème $(S)$ consiste en un système de $N$ équations à $N$ inconnues. Il s'écrit sous forme matricielle
        \[
            AU = B
        \]
        
        avec $A \in M_N(\R)$ et $B \in \R^N$ donnés par :
        \[
            \begin{array}{cc}
                A =
                \begin{pmatrix}
                    2 + h^2 q_1 & -1 + \frac{h}{2}p_1 & & 0\\
                    -1 - \frac{h}{2}p_2 & 2 + h^2 q_2 & -1 + \frac{h}{2}p_2 & \\
                    & \ddots & \ddots & & \\
                    & & 2 + h^2 q_{N-1} & -1 + \frac{h}{2}p_{N-1} \\
                    0 & & -1 - \frac{h}{2}p_N & 2 + h^2 q_N
                \end{pmatrix}
                &
                B =
                \begin{pmatrix}
                    h^2 f_1 + \alpha (1 + \frac{h}{2}p_1) \\
                    h^2 f_2 \\
                    \hdots \\
                    h^2 f_{N-1} \\
                    h^2 f_n + \beta (1 - \frac{h}{2} p_N)
                \end{pmatrix}
            \end{array}
        \]

        Le système $(S)$ possède une unique solution pour $h$ assez petit.
    \end{remark}

    \begin{ftheo}
        Si $q \geq 0$ sur $[a,b]$ et $h < \dfrac{2}{\norm{p}_\infty}$ alors
        $A$ est inversible.
        \label{th:1-pbvp-Ainv}
    \end{ftheo}

On va montrer ce résultat dans le cas où $q > 0$ sur $]a,b[$.

    \begin{fdef}
        $A \in M_N(\C)$ est à diagonale strictement dominante si
        \[
            |a_{ii}| > \sum_{j \ne i} |a_{ij}| \text{ pour tout $i=1,..,N$}
        \]
    \end{fdef}

    \begin{ftheo}
        Une matrice à diagonale strictement dominante est inversible.
        \label{th:1-pbvp-DSD-inv}
    \end{ftheo}

    On vérifie que $A$ est à diagonale strictement dominante sous les hypothèses
    du théorème \ref{th:1-pbvp-Ainv} et pour $q_i>0$. En effet :
    \[
        \left\lbrace
        \begin{array}{c}
            a_{i,i+1} \leq -1 + \frac{h}{2} \norm{p}_\infty < 0 \\
            a_{i,i-1} \leq -1 + \frac{h}{2} \norm{p}_\infty < 0
        \end{array}
        \right.
    \]

    D'où :
    \begin{enumerate}[label=*]
        \item $|a_{11}| - |a_{12}| = 2 + h^2 q_1 - (1 - \frac{h}{2}p_1) \geq
            \underbrace{1 - \frac{h}{2} \norm{p}_\infty}_{>0} + \underbrace{h^2 q_1}_{>0} > 0 $

        \item $|a_{N,N}| - |a_{N,N-1}| = 2 + h^2 q_N - (1 - \frac{h}{2}p_N) \geq
            \underbrace{1 - \frac{h}{2} \norm{p}_\infty}_{>0} + \underbrace{h^2 q_N}_{>0} > 0 $

        \item Pour $i = 2, \dots, N-1$ :
            \begin{align*}
                |a_{ii}| - |a_{i,i+1}| - |a_{i,i-1}| & = 2 + h^2 q_i - (1-\frac{h}{2}p_i) - (1 + \frac{h}{2}p_i) \\
                & = h^2 q_i > 0
            \end{align*}
    \end{enumerate}

    $A$ est donc à diagonale strictement dominante, et donc inversible d'après
    le théorème \ref{th:1-pbvp-DSD-inv}. Cela prouve le théorème \ref{th:1-pbvp-Ainv} dans le cas $q > 0$.

\section{Étude de la convergence dans un cas simplifié}
Nous allons montrer que $u_i \underset{h\to0}{\longrightarrow} u(x_i)$

\[
    (P) \;
    \left\lbrace
    \begin{array}{ccc}
        -u''(x) + c(x) u(x) = f(x) & & \text{sur } ]0,1[ \\
            u(0) = u(1) = 0 & & c(x) \geq 0 \text{ est $\Co^2$ sur $[0,1]$}
    \end{array}
    \right.
\]

Le problème discrétisé s'écrit : ($x_i = ih, c_i = c(x_i), f_i = f(x_i)$)
\[
    (S) \;
    \left\lbrace
    \begin{array}{ccc}
        -\frac{u_{i+1} - 2 u_i + u_{i-1}}{h^2} + c_i u_i = f_i & & i=1,..,N \\
        u_0=u_{N+1}=0
    \end{array}
    \right.
\]

Soit encore pour $U=\tpo (u_1,\cdots,u_N)$, $F = \tpo (f_1, \cdots,f_N)$
\[
    \begin{array}{ccc}
        \dfrac{1}{h^2}A \: U = F,
        & \hfill &
        A =
        \begin{pmatrix}
            c_1 h^2 + 2 & & -1 & & 0 \\
            -1 & & \ddots & \ddots & \\
            & & \ddots & \ddots & -1 \\
            0 & & & -1 & c_N h^2 + 2
        \end{pmatrix}
    \end{array}
\]

On a $A = M + h^2 C$ avec 
\[
    M = 
    \begin{pmatrix}
        2 & -1 & & 0 \\
        -1 & \ddots & & -1 \\
        & & \ddots & & \\
        0 & & -1 & -2
    \end{pmatrix}
    \text{ et } 
    C =
    \begin{pmatrix}
        c_1 & & 0 \\
        & \ddots & \\
        0 & & c_N
    \end{pmatrix}
\]

Nous allons établir quelques propriétés de $M$ qui seront utiles par la suite.

\begin{lemme}
    Les valeurs propres de $M$ sont $\lambda_k = 4 \sin^2 \left[ \dfrac{k\pi}{2(N+1)} \right]$, ($k=1,\cdots,N$), et les vecteurs propres associés
    $V^k = \tpo (v_1^k, \cdots , v_N^k)$ vérifient
    \[
        v_i^k = \alpha \sin \left( \frac{ik\pi}{N+1} \right) \hspace{0.7cm} i=1,\cdots,N
    \]

    ($\alpha \in \R$ constante arbitraire)
    \label{lemme:1-pbvp-lemme1}
\end{lemme}

\begin{preuve}
    $M$ est réelle symétrique donc ses valeurs propres $\lambda$ sont réelles.
    L'équation $MV = \lambda V$ s'écrit :
    \[
        \left\lbrace
        \begin{array}{cc}
            2 v_j - v_{j+1} - v_{j-1} = \lambda v_j & j=1,\cdots,N \\
            v_0 = v_{n+1} = 0
        \end{array}
        \right.
    \]

    Pour $1 \leq j \leq N$, $v_j$ vérifie donc une relation de récurrence linéaire
    du second ordre.
    Les suites solutions de cette relation de récurrence forment un espace
    vectoriel de dimension 2.

    En posant $v_j = ar^j$ on obtient l'équation caractéristique
    $r^2 + (\lambda-2) r + 1 = 0$, de discriminant $\Delta = \lambda(\lambda - 4)$

    \begin{enumerate}[label=•]
        \item Plaçons-nous d'abord dans le cas $\Delta < 0$, c'est-à-dire
        $\lambda \in ]0,4[$. $r$ est complexe et de module 1, on pose $r = e^{iq}$
            et on obtient $4 \sin^2 \left(\frac{q}{2}\right) = \lambda$.
            Les solutions de la relation de récurrence s'écrivent
            $v_j = a e^{iqj} + \overline{a} e^{-iqj} \hspace{10pt} (a \in \C, q \in [0,\pi])$

            Soit encore $v_j = A \sin(qj) + B \cos(qj)$.

            On cherche maintenant $q,A,B$ de manière à satisfaire les conditions aux limites.
            $v_0 = 0$ donne $B = 0$ et $v_{N+1} = 0$ donne $q = \dfrac{k\pi}{N+1}, k = 1,\cdots,N$. D'où :
            \[
                \lambda = 4 \sin^2 \left( \frac{k \pi}{2(N+1)} \right),
                \hspace{0.7cm} v_j = A\sin\left( \frac{k\pi}{N+1}j \right)
            \]

        \item L'étude précédente donne $N$ valeurs propres distinctes de $M$.
            On a donc trouvé toutes ses valeurs propres et il n'est pas
            nécessaire de considérer le cas $\Delta \geq 0$ (on n'obtiendrait
            alors que la solution $V = 0$).
    \end{enumerate}
\end{preuve}

\begin{lemme}[Inégalité de Poincaré discrète]
    \[
        \frac{1}{h^2} \tpo X \; M \; X \geq 4 \norm{X}_2^2 \hspace{1cm} \forall X \in \R^N, h = \frac{1}{N+1}
    \]
    \label{lemme:1-pbvp-lemme2}
\end{lemme}

\begin{preuve}
    On a $M = P \wedge P^{-1}$ où
    $\wedge =
    \begin{pmatrix}
        \lambda_1 & & 0 \\
        & \ddots & \\
        0 & & \lambda_N
    \end{pmatrix}
    $
    et
    $ P =
    \begin{pmatrix}
        V1 & \cdots & V^N
    \end{pmatrix}
    $

    ($V^k$ calculés dans le lemme \ref{lemme:1-pbvp-lemme1}). $M$ est symétrique
    et les valeurs propres $\lambda_k$ sont distinctes donc les vecteurs propres
    $V^k$ sont deux à deux orthogonaux.

    En fixant $\alpha = \sqrt{\frac{2}{N+1}}$ on a $\norm{V^k}_2 = 1$ donc
    $\tpo P P = I$.

    En posant $Y = \tpo P X$ on obtient 
    \[
        \frac{1}{h^2} \tpo X \, M \, X =
        \frac{1}{h^2} \tpo Y \, \wedge \, Y = \sum_{k=1}^N \lambda_k \: y_k^2
        \; \geq \; \frac{\lambda_1}{h^2} \norm{Y}_2^2 = \frac{\lambda_1}{h^2} \norm{X}_2^2
    \]

    Donc : 
    \[
        \frac{1}{h^2} \: \frac{\tpo X \, M \, X}{\norm{X}_2^2} \geq \frac{\lambda_1}{h^2} = \frac{1}{h^2} 4 \sin^2 \left( \frac{\pi k }{2} \right) \geq 4
    \]
    \[
        ( \sin x \geq \frac{2}{\pi} x \text{ si } 0 \leq x \leq \dfrac{\pi}{2},
        \text{ et } \frac{\pi h}{2} \leq \frac{\pi}{4} )
    \]
\end{preuve}

    On en déduit l'estimation suivante sur la norme euclidienne de la solution
    du système linéaire :
\begin{lemme}
    La solution de $\dfrac{1}{h^2} \, A \, W = B$ vérifie :
    \[
        \norm{W}_2 \leq \frac{1}{4} \norm{B}_2
    \]
    \label{lemme:1-pbvp-lemme3}
\end{lemme}

\begin{preuve}
    \[
        \frac{1}{h^2} \tpo W \, A \, W = \frac{1}{h^2} \tpo W \, M \, W +
        \underbrace{\tpo W \, C \, W}_{\geq 0} \geq 4 \norm{W}_2^2
        % \underbrace{\tpo W \, C \, W}_{\geq 0} \geq 4 \norm{W}_2^2
    \] d'après le lemme \ref{lemme:1-pbvp-lemme2}

    Par ailleurs :
    \[
        \frac{1}{h^2} \tpo W \, A \, W = \tpo W \, B \leq \norm{W}_2 \norm{B}_2
        \hspace{1cm} \text{ (inégalité de Cauchy-Schwartz) }
    \]

    Donc $4 \norm{W}_2^2 \leq \norm{W}_2 \norm{B}_2$
\end{preuve}

Ce résultat correspond à la \textbf{``stabilité $L^2$ du schéma numérique''}.

En effet, définissons la norme : $\norm{F}_h = \sqrt{h} \norm{F}_2$.

Si $f \in \Co^0([0,1]), $ 
\[
    \norm{F}_h^2 = \sum_{i=1}^N f(ih)^2 \times h \underset{h\to0}{\longrightarrow} \int_0^1 f^2 \dx \hspace{0.7cm} \text{(somme de Riemann)}
\]

C'est-à-dire $\norm{F}_h \underset{h\to 0}{\longrightarrow} \norm{f}_{L^2(0,1)}$

Le lemme \ref{lemme:1-pbvp-lemme3} donne $\norm{U}_h \leq \frac{1}{4}\norm{F}_h \leq \frac{1}{2} \norm{f}_{L^2}$ pour $h$ assez petit.

Soit $U_h$ la ``solution numérique'' définie sur [0,1] par interpolation
linéaire des $u_i$ (donc $U_h(x_i) = u_i$). On vérifie que $\norm{U_h}_{L^2} = \norm{U}_h$.

On a donc 
\[
    \norm{U_h}_{L^2} \leq \frac{1}{2} \norm{f}_{L^2} \hspace{0.7cm} \text{pour $f$ fixé et $h$ assez petit}
\]

La norme $L^2$ de la solution numérique reste bornée quand $h \longrightarrow 0$
et on dit alors que le schéma est stable dans $L^2$.

Notons maintenant $\tilde u_i = u(x_i)$ et $\tilde U = \tpo (\tilde u_1, \cdots , \tilde u_N )$

À l'aide d'un développement de Taylor en $x_i$, on obtient :
\begin{align*}
    & \frac{\tilde u_{i+1} - 2 \tilde u_i + \tilde u_{i-1}}{h^2} + c_i \tilde u_i - f_i \\
    = & - u''(x_i) + c(x_i) u(x_i) - f(x_i) + o(1) \hspace{1cm} \text{quand } h\to0 \\
    = & \;o(1)
\end{align*}

Si $u \in \Co^4([0,1])$ (càd pour $f \in \Co^2([0,1])$) on a mieux :
\[
    - \frac{\tilde u_{i+1} - 2 \tilde u_i + \tilde u_{i-1}}{h^2} + c_i \tilde u_i - f_i = - \frac{h^2}{24} u^{(4)} (\theta_i^+) - \frac{h^2}{24} u^{(4)}(\theta_i^-)
\] 
    \hfill $\Big(\theta_i^+ \in ]x_i,x_{i+1}[, \text{ et } \theta_i^- \in ]x_{i-1},x_i[ \Big)$

    D'où :
    \[
        \left| - \frac{\tilde u_{i+1} - 2 \tilde u_i + \tilde u_{i-1}}{h^2}
        + c_i \tilde u_i - f_i \right| \leq \frac{h^2}{12} \norm{U^{(4)}}_\infty
    \]
    \hfill (converge en $h^2$)

    Donc la solution de $(P)$ vérifie $(S)$ avec une erreur qui tend vers 0 quand
    $h \to 0$. Le schéma $(S)$ est donc \textbf{consistant}. Lorsque $u$
    est suffisamment régulière, cette erreur est en $\mathcal{O}(h^2)$;
    le schéma $(S)$ est donc \textbf{d'ordre 2}.

    Les calculs précédents montrent que :
    \[
        \frac{1}{h^2} \, A \tilde U = F + E
    \]
    \hfill avec $\norm{E}_\infty \leq \frac{h^2}{12} \norm{U^{(4)}}_\infty$ si
    $u \in \Co^4([0,1])$

    Puisque $\norm{.}_h \leq \norm{.}_\infty$ on a donc lorsque $h \to 0$ :
    \[
        \norm{E}_h = \mathcal{O}(h^2) \hspace{0.7cm} \text{ si $u$ est $\Co^4$ (càd $f$ est $\Co^2$) }
    \]

    Après avoir montré la stabilité et la consistance du schéma $(S)$, on peut
    maintenant conclure sur sa convergence.

    En effet :
    \[
        \frac{1}{h^2} \, A(\tilde U - U) = E
    \]

    amène grâce au lemme \ref{lemme:1-pbvp-lemme3} :
    \[
        \norm{\tilde U - U}_h \leq \frac{1}{4} \norm{E}_h = \mathcal O(h^2)
        \hspace{0.5cm} \text{ si $f$ est $\Co^2$}
    \]

    \begin{remark}
        \begin{enumerate}[label=•]
            \item Si on a simplement $f \in \Co^0([0,1])$, on peut montrer
                que $\norm{\tilde U - U}_h \longrightarrow 0$ lorsque $h \to 0$
                à l'aide d'un argument de densité (cf TD)

            \item Si $f$ est $\Co^2$, $\norm{\tilde U - U}_\infty \leq \norm{\tilde U - U}_2 = \frac{1}{\sqrt h} \norm{\tilde U - U}_h$ donne
                $\norm{\tilde U - U}_\infty = \mathcal O (h^{3/2}))$

                Nous verrons en TD comment obtenir un résultat de convergence
                plus fort : $\norm{\tilde U - U}_\infty = \mathcal O (h^2)$.
                Le principe est le même (stabilité + consistance en $\norm{.}_\infty \implies $ convergence)
                mais la stabilité s'obtient d'une autre manière (en utilisant
                le ``principe du maximum'').
        \end{enumerate}
    \end{remark}

\section{Conclusion}

La méthode des différences finies conduit à la résolution de
\textbf{systèmes linéaires de grande taille} pour lesquels nous introduirons
différents types d'outils. Bien sûr nous n'avons vu qu'une introduction
aux différences finies. Cette méthode est beaucoup utilisée pour la résolution
d'équation aux dérivées partielles, comme l'équation de la chaleur
${\derpart{u}{t} = \derpart{^2u}{x^2}}$ ou l'équation de Poisson 
${-\left( \derpart{^2 \phi}{x^2} + \derpart{^2 \phi}{y^2} \right) = f(x,y)}$

\subsection*{Exemples de schémas pour ces deux équations}

\begin{enumerate}[label=•]
    \item \underline{Équation de la chaleur :}
        \[
            \frac{U_j^{n+1}- U_j^n}{\Delta t} = \frac{U_{j+1}^{n+1} - 2 U_j^{n+1} + U_{j-1}^{n+1}}{\Delta x^2}
        \]

        $U_j^n \simeq U(j\Delta x, n\Delta t) \hspace{1cm} U_j^0 = U(j \Delta x,0)$

        Schéma \textbf{implicite} : $(U_j^{n+1})_j$ s'obtient à partir de $(U_j^n)_j$
        en résolvant un système linéaire.

    \item \underline{Équation de Poisson :}
        \[
            \frac{- \phi_{i+1,k} - \phi_{i,k+1} + 4 \phi_{i,k} - \phi_{i,k-1} - \phi_{i-1,?}}{h^2} = f_{i,k}
        \]

        $\phi_{i,k} \simeq \phi(ih,kh)$

        $(\phi_{i,k})_{i,k}$ s'obtient en fonction de $(f_{i,k})_{i,k}$ en
        résolvant un système linéaire.
\end{enumerate}
